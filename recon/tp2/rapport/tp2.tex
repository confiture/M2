\documentclass[11pt,a4paper]{article}

\usepackage{pdfpages}
\usepackage{amsfonts}
\usepackage{amsmath}
\usepackage{amssymb}
\usepackage{amsthm}
\usepackage[francais]{babel}
\usepackage[dvips,lmargin=3cm,rmargin=3cm,tmargin=3cm,bmargin=4cm]{geometry}
\usepackage{graphicx}
\usepackage{hyperref}
\usepackage[utf8]{inputenc}
\usepackage{listings}
\usepackage{textcomp}
\usepackage{xcolor}
\usepackage{float}
\usepackage[T1]{fontenc}
%\usepackage[nottoc, notlof, notlot]{tocbibind}
\usepackage{graphicx}

\usepackage{subfig}

% Macros pratique pour les maths.
\newcommand{\bs}{\symbol{92}}
\newcommand{\Z}{\mathbb{Z}}
\newcommand{\Q}{\mathbb{Q}}
\newcommand{\N}{\mathbb{N}}
\newcommand{\R}{\mathbb{R}}
\newcommand{\C}{\mathbb{C}}
\newcommand{\K}{\mathbb{K}}
\newcommand{\B}{\mathcal{B}}
\newcommand{\tab}{\hspace*{1cm}}
\newcommand{\norm}[1]{\left\vert \left\vert {#1}  \right\vert \right\vert}
\newcommand{\abs}[1]{ \left\vert {#1}  \right\vert}
\newcommand{\vect}[1]{\overrightarrow{#1}}
\newcommand{\scal}[2]{\langle #1 \vert #2 \rangle}
\newcommand{\exems}{$ $ \\ \noindent{{\textbf{Exemples.~}}}}
\newcommand{\exem}{$ $ \\ \noindent{{\textbf{Exemple.~}}}}
\newcommand{\rem}{$ $\\\ \noindent{{\textbf{Remarque.~}}}}
\newcommand{\rems}{$ $ \\ \noindent{{\textbf{Remarques.~}}}}
\newcommand{\rap}{$ $ \\ \noindent{{\textbf{Rappel.~}}}}
\newcommand{\car}[1]{|#1|}
\newcommand{\card}{\mbox{card}}
\newcommand{\stab}{\mbox{Stab}}
\newcommand{\ba}[1]{\overline{#1}}
\newcommand{\inte}[1]{ [\![{#1} ]\!]}

% Configuration du package lstlisting pour du C++.
% \lstset{
%   basicstyle=\small,
%   breaklines=true,
%   commentstyle=\color[rgb]{0.3,0.3,0.3}\textit,
%   emphstyle=\rmfamily\color{blue},
%   frame=single,
%   keywordstyle=\color{blue},
%   language=C++,
%   lineskip=0pt,
%   numbers=left,
%   numbersep=5pt,
%   numberstyle=\tt\small\color{gray},
%   showstringspaces=false,
%   tabsize=4,
%   texcl=true
% }
\definecolor{colKeys}{rgb}{0,0,1}
\definecolor{colIdentifier}{rgb}{0,0,0}
\definecolor{colComments}{rgb}{0,0.5,1}
\definecolor{colString}{rgb}{0.6,0.1,0.1}

\lstset{%configuration de listings
float=hbp, %
basicstyle=\ttfamily \small, %
identifierstyle=\color{colIdentifier}, %
keywordstyle=\color{colKeys}, %
stringstyle=\color{colString}, %
commentstyle=\color{colComments}, %
columns=flexible, %
tabsize=4, %
frame=trBL, %
frameround=tttt, %
extendedchars=true, %
showspaces=false, %
showstringspaces=false, %
numbers=left, %
numberstyle=\tiny, %
breaklines=true, %
breakautoindent=true, %
captionpos=b, %
xrightmargin=0cm, %
xleftmargin=0cm,
language=scilab,
commentstyle=\textit
}



\title{
  \huge{\bf TP2 - Reconstruction de surfaces}
}

\author{
    \textsc{Trlin} Moreno \\
}

\date{25 janvier 2011}

\begin{document}

 \maketitle
  \begin{center}
   \includegraphics[height=4cm]{../images/imag.png}

  \end{center}
 \tableofcontents

\section{EXERCICE-SM 1}
\subsection{Résultats}

\begin{tabular}{|c|c|}
\hline
\includegraphics[width=8cm]{../images/SM1-init.png} & \includegraphics[width=8cm]{../images/SM1-dp.png} \\
Surface initiale                         & Dérivée première modifiée. \\
\hline
\includegraphics[width=8cm]{../images/SM1-init.png} & \includegraphics[width=8cm]{../images/SM1-ds.png} \\
Surface initiale                         & Twists légèrement modifiés. \\
\hline
\includegraphics[width=8cm]{../images/SM1-init.png} & \includegraphics[width=8cm]{../images/SM1-dsG.png} \\
Surface initiale                         & Grande modification des twists. \\
\hline
\end{tabular}

\paragraph{Observations :} On voit que la modification de la dérivée première conduit à une modification de
la surface. Une petite modification des twists ne change pas beaucoup la surface. Une grande modification
des twists tord la surface.

\section{EXERCICE-SM 2}
\subsection{Le code}

\begin{lstlisting}
function [SX,SY,SZ,X1,Y1,Z1] = ...
  interp_prod_tensoriel2(X,Y,Z,u,v,Su,Sv,methode_u,methode_v)

  // verification du nombre d'arguments
  nargin = argn(2);
  if nargin<7
    error('syntaxe [SX,SY,SZ] = interp_prod_tensoriel(X,Y,Z,u,v,Su,Sv[,methode_u[,methode_v]])');
  end
  if nargin<8
    methode_u = 'natural';
  end
  if nargin<9
    methode_v = methode_u;
  end

  // verification des arguments
  M = length(u);
  N = length(v);
  P = length(Su);
  Q = length(Sv);
  if (M<2 | N<2 | P<2 | Q<2)
    error('les vecteurs u, v, Su, Sv doivent avoir au moins 2 valeurs');
  end
  if (min(diff(u))<=0 | min(diff(v))<=0 | min(diff(Su))<=0 | min(diff(Sv))<=0)
    error('les vecteurs u, v, Su, Sv doivent etre composees de valeurs strictement croissantes');
  end

  if (size(X,1)~=N | size(X,2)~=M | ...
      size(Y,1)~=N | size(Y,2)~=M | ...
      size(Z,1)~=N | size(Z,2)~=M)
    error('les tableaux (X,Y,Z) doivent avoir des dimensions correspondant a (u,v)');
  end

  // mettre u et Su en ligne, v et Sv en colonne
  u  = u(:)' ;
  Su = Su(:)';
  v  = v(:)  ;
  Sv = Sv(:) ;

  ////// interpolation 1-D suivant le parametre v
  // selection de la routine Scilab suivant la methode
  select methode_v
    case 'not_a_knot'
      routine_interp = 'splin1'; methode_interp = '''not_a_knot''';
    case 'natural'
      routine_interp = 'splin1'; methode_interp = '''natural''';
    case 'periodic'
      routine_interp = 'splin1'; methode_interp = '''periodic''';
    case 'fast'
      routine_interp = 'splin1'; methode_interp = '''fast''';
    case 'fast_periodic'
      routine_interp = 'splin1'; methode_interp = '''fast_periodic''';
    else
      routine_interp = 'splin1'; methode_interp = '''natural''';
  end

  X1 = zeros(Q,M);
  Y1 = zeros(Q,M);
  Z1 = zeros(Q,M);
  s_interp_X = ['X1(:,i) = '+ routine_interp+ '(v, X(:,i), Sv,'+ methode_interp+ ');'];
  s_interp_Y = ['Y1(:,i) = '+ routine_interp+ '(v, Y(:,i), Sv,'+ methode_interp+ ');'];
  s_interp_Z = ['Z1(:,i) = '+ routine_interp+ '(v, Z(:,i), Sv,'+ methode_interp+ ');'];
  for i=1:M
    execstr([s_interp_X;s_interp_Y;s_interp_Z]);
  end

  ////// interpolation 1-D suivant le parametre u
  // selection de la routine Scilab suivant la methode
  select methode_u
    case 'not_a_knot'
      routine_interp = 'splin1'; methode_interp = '''not_a_knot''';
    case 'natural'
      routine_interp = 'splin1'; methode_interp = '''natural''';
    case 'periodic'
      routine_interp = 'splin1'; methode_interp = '''periodic''';
    case 'fast'
      routine_interp = 'splin1'; methode_interp = '''fast''';
    case 'fast_periodic'
      routine_interp = 'splin1'; methode_interp = '''fast_periodic''';
    else
      routine_interp = 'splin1'; methode_interp = '''natural''';
  end

  s_interp_X = ['SX(i,:) = '+ routine_interp+ '(u, X1(i,:), Su,'+ methode_interp+ ');'];
  s_interp_Y = ['SY(i,:) = '+ routine_interp+ '(u, Y1(i,:), Su,'+ methode_interp+ ');'];
  s_interp_Z = ['SZ(i,:) = '+ routine_interp+ '(u, Z1(i,:), Su,'+ methode_interp+ ');'];
  for i=1:Q
    execstr([s_interp_X;s_interp_Y;s_interp_Z]);
  end

endfunction
\end{lstlisting}

\subsection{Résultats}

\begin{tabular}{|c|c|}
\hline
$u$ puis $v$                                        & $v$ puis $u$\\
\hline
\includegraphics[width=8cm]{../images/SM2-uv-i.png} & \includegraphics[width=8cm]{../images/SM2-vu-i.png} \\
\includegraphics[width=8cm]{../images/SM2-uv-ii-1.png} & \includegraphics[width=8cm]{../images/SM2-vu-ii-1.png} \\
\includegraphics[width=8cm]{../images/SM2-uv-ii-2.png} & \includegraphics[width=8cm]{../images/SM2-vu-ii-2.png} \\
\hline
\end{tabular}

\paragraph{Observations :} Les interpolations sont bien les mêmes.\\ \\

Les différents schémas d'interpolation :\\
\begin{tabular}{|c|c|}
\hline
\includegraphics[width=8cm]{../images/SM2-notknot.png} & \includegraphics[width=8cm]{../images/SM2-fast.png} \\
not\_a\_knot                                             & fast \\
\hline
\includegraphics[width=8cm]{../images/SM2-uPervnat.png} & \includegraphics[width=8cm]{../images/SM2-fast_per.png} \\
periodic en $u$ et natural en $v$                       & fast\_periodic \\
\hline
\end{tabular}

\begin{center}
\begin{tabular}{|c|}
\hline
\includegraphics[width=8cm]{../images/SM2-fast.png} \\
natural \\
\hline
\end{tabular}
\end{center}


\paragraph{Observations :} Les interpolations fast et natural sont quasiment les mêmes : elles interpolent
le polygone de contrôle avec une forme plutôt quadrangulaire et elles possèdent une discontinuité en la dérivée
première. Les autres interpolations sont régulières. L'interpolation not\_a\_knot a une forme plutôt triangulaire.
L'interpolation periodic en $u$ et natural $v$ a une forme circulaire. L'interpolation fast\_periodic est comme
l'interpolation fast (ou natural) sans la discontinuité en la dérivée première.

 \pagebreak
%%%%%%%%%%%%%%%%%%%%%%%%%%%%%%
 \end{document}
