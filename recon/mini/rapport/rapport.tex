\documentclass[11pt,a4paper]{article}

\usepackage{pdfpages}
\usepackage{amsfonts}
\usepackage{amsmath}
\usepackage{amssymb}
\usepackage{amsthm}
\usepackage[francais]{babel}
\usepackage[dvips,lmargin=3cm,rmargin=3cm,tmargin=3cm,bmargin=4cm]{geometry}
\usepackage{graphicx}
\usepackage{hyperref}
\usepackage[utf8]{inputenc}
\usepackage{listings}
\usepackage{textcomp}
\usepackage{xcolor}
\usepackage{float}
\usepackage[T1]{fontenc}
%\usepackage[nottoc, notlof, notlot]{tocbibind}
\usepackage{graphicx}

\usepackage{subfig}

% Macros pratique pour les maths.
\newcommand{\bs}{\symbol{92}}
\newcommand{\Z}{\mathbb{Z}}
\newcommand{\Q}{\mathbb{Q}}
\newcommand{\N}{\mathbb{N}}
\newcommand{\R}{\mathbb{R}}
\newcommand{\C}{\mathbb{C}}
\newcommand{\K}{\mathbb{K}}
\newcommand{\B}{\mathcal{B}}
\newcommand{\tab}{\hspace*{1cm}}
\newcommand{\norm}[1]{\left\vert \left\vert {#1}  \right\vert \right\vert}
\newcommand{\abs}[1]{ \left\vert {#1}  \right\vert}
\newcommand{\vect}[1]{\overrightarrow{#1}}
\newcommand{\scal}[2]{\langle #1 \vert #2 \rangle}
\newcommand{\exems}{$ $ \\ \noindent{{\textbf{Exemples.~}}}}
\newcommand{\exem}{$ $ \\ \noindent{{\textbf{Exemple.~}}}}
\newcommand{\rem}{$ $\\\ \noindent{{\textbf{Remarque.~}}}}
\newcommand{\rems}{$ $ \\ \noindent{{\textbf{Remarques.~}}}}
\newcommand{\rap}{$ $ \\ \noindent{{\textbf{Rappel.~}}}}
\newcommand{\car}[1]{|#1|}
\newcommand{\card}{\mbox{card}}
\newcommand{\stab}{\mbox{Stab}}
\newcommand{\ba}[1]{\overline{#1}}
\newcommand{\inte}[1]{ [\![{#1} ]\!]}

% Configuration du package lstlisting pour du C++.
% \lstset{
%   basicstyle=\small,
%   breaklines=true,
%   commentstyle=\color[rgb]{0.3,0.3,0.3}\textit,
%   emphstyle=\rmfamily\color{blue},
%   frame=single,
%   keywordstyle=\color{blue},
%   language=C++,
%   lineskip=0pt,
%   numbers=left,
%   numbersep=5pt,
%   numberstyle=\tt\small\color{gray},
%   showstringspaces=false,
%   tabsize=4,
%   texcl=true
% }
\definecolor{colKeys}{rgb}{0,0,1}
\definecolor{colIdentifier}{rgb}{0,0,0}
\definecolor{colComments}{rgb}{0,0.5,1}
\definecolor{colString}{rgb}{0.6,0.1,0.1}

\lstset{%configuration de listings
float=hbp, %
basicstyle=\ttfamily \small, %
identifierstyle=\color{colIdentifier}, %
keywordstyle=\color{colKeys}, %
stringstyle=\color{colString}, %
commentstyle=\color{colComments}, %
columns=flexible, %
tabsize=4, %
frame=trBL, %
frameround=tttt, %
extendedchars=true, %
showspaces=false, %
showstringspaces=false, %
numbers=left, %
numberstyle=\tiny, %
breaklines=true, %
breakautoindent=true, %
captionpos=b, %
xrightmargin=0cm, %
xleftmargin=0cm,
language=scilab,
commentstyle=\textit
}



\title{
  \huge{\bf Mini-projet de reconstruction}
}

\author{
    \textsc{Trlin} Moreno \\
}

\date{6 mars 2011}

\begin{document}

 \maketitle
  \begin{center}
   \includegraphics[height=4cm]{../images/imag.png}

  \end{center}
 \tableofcontents

\pagebreak

\section{Partie courbes}
\subsection{Le programme}
\subsubsection{Structure générale du programme}
En plus des fichiers sources fournis, le programme est aussi composé des fichiers courbe.cpp et courbe.hpp.

Ces deux fichiers dénifissent une structure {\tt Point3} (point de 3 coordonnées) et une structure {\tt grille} qui est une grille régulière en deux dimensions.
Cette grille n'est pas indicée en les valeurs réelles des points mais en les indices du tableau qui définit cette grille.
Je me sers de cette grille pour fabriquer la grille régulière des distances signées ({\tt calc\_grille\_dist}) puis je calcule la courbe
iso-valeur ({\tt calc\_iso\_courbe}) pour reconstruire la courbe issue des points. Il y a des points à trois coordonnées car je me sers
de la troisième coordonnée pour stocker la distance signée en $(x,y)$ du point en question. Lorsque je calcule la courbe iso-valeur, je calcule l'intersection
du plan $z=0$ avec la surface définie par les points à 3 coordonnées.

Pour ce qui est du calcul des normales orientées ({\tt normales}), je me sers des structures fournies {\tt Graph}, {\tt Arbre} et {\tt Arete} ainsi que
des fonctions définies dessus, plus particulièrment {\tt calcule\_ACM} qui calcule l'arbre couvrant minimal.

\subsubsection{Structure {\tt grille}}
La grille est principalement constituée d'un tableau à deux dimension de doubles. Elle est constituée néanmoins des champs :
\begin{itemize}
\item {\tt int nl} : le nombre de lignes de la grille régulière
\item {\tt int nc} : le nombre de colonnes de la grille régulière
\item {\tt double pas} : le pas de la grille régulière
\item {\tt double xmin,xmax,ymin,ymax,zmin,zmax} : minimum des abscisses, maximum des abscisses, minimum des ordonnées, maximum des ordonnées
\item {\tt double** vals} : tableau contenant les valeurs de la grille régulière
\end{itemize}

Les points de cette grille sont donc espacés de {\tt pas}, et on peut associer à chaque point de la grille une valeur {\tt double}.
Cette grille n'est pas indicée sur les vrais indices des points mais sur les indices du tableau {\tt vals} contenant les valeurs en les points
de la grille. L'indice $(i,j)$ de la grille correspond en réalité au point $(j*pas+xmin,i*pas+ymin)$.

Cette grille est utilisé pour calculer la grille régulière des distances signées, et à partir de cette grille, on calcule la courbe iso-valeur.

La principale fonction de cette structure est en fait de contenir beaucoup de champs.
Cela évite de passer trop d'arguments aux fonctions.

\subsubsection{La fonction {\tt calc\_grille\_dist}}
Cette fonction calcule la grille régulière des distances signées à partir des points, des normales et du pas qu'on lui passe en paramètre.

Les bords de la grille régulière ({\tt xmin, xmax, ymin, ymax}) sont calculés avec la fonction {\tt boite\_enlgobante}. Mais ces bords sont
repoussés de deux fois le pas afin d'éviter les problèmes sur les bords de la grille. Si {\tt xmin, xmax, ymin, ymax} est le résultat de
la fonction {\tt boite\_enlgobante} alors $$xmin_{grille}=xmin-2*pas,~xmax_{grille}=xmax+2*pas$$
$$~ymin_{grille}=ymin-2*pas,~ ymax_{grille}=ymax+2*pas $$

Lors de la construction de la grille régulière des distances signées, il faut trouver pour chaque point le point de la grille le plus proche.
Voici la méthode que j'ai utilisé :\\
supposons que nous cherchons les indices $(i,j)$ de la grille représentant le point le plus proche du point $(x,y)$ en entrée. Soit $E$ la fonction
partie entière.\\ Si $y-(E((y-ymin)/pas))*pas-ymin<=pas/2.0$ alors $i=E((y-ymin)/pas)$ sinon $i=E((y-ymin)/pas)+1$ de même pour $j$.

\subsubsection{La méthode {\tt calc\_iso\_courbe}}
Cette méthode calcule la courbe iso-valeur à partir de la grille régulière des distances signées. L'idée de cette méthode est simple, on parcourt
toutes les mailles de la grille. Chaque maille est décomposée en deux triangles. Les points des triangles sont en dimension 3. La troisième dimension
étant la valeur de la grille en $(i,j)$, c'est à dire la distance signée. L'intersection de chaque triangle avec le plan $z=0$ est calculée si elle existe.
À la suite de cette méthode, nous avons une liste de segments définissant la courbe iso-valeur.

\subsubsection{La fonction {\tt normales}}
La fonction {\tt normales} calcule les normales en les points donnés en fonction d'un rayon $r$. Pour chaque point $P_i$, on considère tous l'ensemble $E$
les points dont la à distance de $P_i$ est inférieure ou égale à $r$. À partir de $E$, on calcule la droite aux moindres carrés grâce à la fonction {\tt droite\_mc}, à partir
de ça, on récupère le vecteur normal au vecteur directeur de la droite qui est aussi le vecteur normale au point $P_i$. Après cette étape, il faut orienter les normales.
Pour cela, on construit le graphe $G$ définit ainsi : \\
Arête $e \in G $ ssi $d(P_i,P_j) \leq r$ et $v(e)=1-|\vec{n_i}.\vec{n_j}|$. Les noeuds de $G$ sont donc les numéros de normales.
On construit ensuite l'arbre couvrant minimal $A$ à partir de $G$. On lance ensuite la fonction {\tt oriente\_normales} sur l'arbre couvrant minimal.

\subsubsection{La fonction {\tt oriente\_normales}}
Cette fonction est récursive. Elle prend en entrée l'arbre couvrant minimal $A$ et le tableau des normales ainsi que le numéro de noeud à traiter.
On parcourt récursivement $A$ : pour chaque noeud (numéro de normale), on oriente les normales voisines de la même façon que la normale courante, puis on
relance cette fonction sur les voisins.

\subsection{Manuel utilisateur}
Le programme guide entièrement l'utilisateur. Le programme demandera le facteur multiplicatif pour le pas de la grille
$g=c_g*Max_i(Min_{i \neq j}(d(P_i,P_j))$. Le facteur multiplicatif est donc $c_g$.
Le programme peut aussi demander le facteur du bruit $b=c_b*min_{i \neq j}(d(P_i,P_j))$ qui est donc $c_b$. Soit $bruit$ la fonction
qui bruite une jeu de données $P_i$. Lorsqu'on bruite un jeu de données $P_i$ de $b$, cela signifie que $bruit(P_i,b) \in [P_i-b, P_i+b]$.
Le programme peut aussi demander le facteur multiplicatif du rayon $r=c_r*g$ impliqué dans le calcul des normales. Ce facteur est donc $c_r$.

\subsection{Les résultats}
\subsubsection{Avec les normales orientées}
On peut observer ici les variations de $g$, le pas de la grille. On prend g comme $$g=c*Max_i(Min_{i \neq j}(d(P_i,P_j))$$

\begin{center}
\begin{tabular}{|c|c|}
\hline
c=1         & c=1.3\\
\hline
\hline
\includegraphics[width=8cm]{../images/partie1-jeu1-fe10.jpg} & \includegraphics[width=8cm]{../images/partie1-jeu1-fe13.jpg} \\
\hline
\includegraphics[width=8cm]{../images/partie1-jeu2-fe10.jpg} & \includegraphics[width=8cm]{../images/partie1-jeu2-fe13.jpg} \\
\hline
\end{tabular}

\begin{tabular}{|c|c|}
\hline
c=1.6         & c=2.8\\
\hline
\hline
\includegraphics[width=8cm]{../images/partie1-jeu1-fe16.jpg} & \includegraphics[width=8cm]{../images/partie1-jeu1-fe28.jpg}\\
\hline
\includegraphics[width=8cm]{../images/partie1-jeu2-fe16.jpg} & \includegraphics[width=8cm]{../images/partie1-jeu2-fe28.jpg}\\
\hline
\end{tabular}
\end{center}
\paragraph{Observations :} Plus $g$ est grand, plus la grille régulière est grossière et plus la courbe est composée de
grands segments.

\subsubsection{Sans les normales orientées}
On fait varier ici le rayon $r$ impliqué comme le calcul des normales. On prend r comme $r=c*g$ où g est le pas. Ici
$$g=Max_i(Min_{i \neq j}(d(P_i,P_j))$$

\begin{center}
\begin{tabular}{|c|c|}
\hline
c=1         & c=1.3\\
\hline
\hline
\includegraphics[width=8cm]{../images/partie2-jeu1-fe10-fr10.jpg} & \includegraphics[width=8cm]{../images/partie2-jeu1-fe10-fr13.jpg}\\
\hline
\includegraphics[width=8cm]{../images/partie2-jeu2-fe10-fr10.jpg} & \includegraphics[width=8cm]{../images/partie2-jeu2-fe10-fr13.jpg}\\
\hline
\end{tabular}

\begin{tabular}{|c|c|}
\hline
c=1.6  & c=2.8\\
\hline
\hline
\includegraphics[width=8cm]{../images/partie2-jeu1-fe10-fr16.jpg} & \includegraphics[width=8cm]{../images/partie2-jeu1-fe10-fr28.jpg}\\
\hline
\includegraphics[width=8cm]{../images/partie2-jeu2-fe10-fr16.jpg} & \includegraphics[width=8cm]{../images/partie2-jeu2-fe10-fr28.jpg}\\
\hline
\end{tabular}
\end{center}

\paragraph{Observations :} On voit sur la courbe ouverte que si $r$ n'est pas assez grand, lorsque les points sont éloignés, ils ne sont
pas ralliés. Ceci se produit car dans ce jeu de données, les points ne sont pas uniformément répartis, si le rayon est trop petit,
les normales aux endroits où les points sont très espacés sont mal calculées car ces points n'ont pas assez de voisins (à cause du petit rayon $r$)
pour calculer une droite aux moindres carrés précise.

\subsubsection{Variation du bruitage et du rayon $r$ sans les normales orientées}
On fait varier ici le rayon $r$ impliqué dans le calcul des normales pour différents bruits $b$. On prend r comme $r=c_r*g$ où g est le pas. Ici
$g=Max_i(Min_{i \neq j}(d(P_i,P_j))$. On prend $b=c_b*min_{i \neq j}(d(P_i,P_j))$.

\pagebreak

Ici le bruit $b=0.2*min_{i \neq j}(d(P_i,P_j))$
\begin{center}
\begin{tabular}{|c|c|}
\hline
\includegraphics[width=8cm]{../imagesCB/partie2-jeu1-fr10-bruit2.jpg} & \includegraphics[width=8cm]{../imagesCB/partie2-jeu1-fr11-bruit2.jpg} \\
$c_r=1.0$                                                             & $c_r=1.1$ \\
\hline
\includegraphics[width=8cm]{../imagesCB/partie2-jeu1-fr30-bruit2.jpg} & \includegraphics[width=8cm]{../imagesCB/partie2-jeu1-fr40-bruit2.jpg} \\
$c_r=3.0$                                                             & $c_r=4.0$ \\
\hline
\end{tabular}
\end{center}

\pagebreak

Ici le bruit $b=0.4*min_{i \neq j}(d(P_i,P_j))$
\begin{center}
\begin{tabular}{|c|c|}
\hline
\includegraphics[width=8cm]{../imagesCB/partie2-jeu1-fr10-bruit4.jpg} & \includegraphics[width=8cm]{../imagesCB/partie2-jeu1-fr11-bruit4.jpg} \\
$c_r=1.0$                                                             & $c_r=1.1$ \\
\hline
\includegraphics[width=8cm]{../imagesCB/partie2-jeu1-fr30-bruit4.jpg} & \includegraphics[width=8cm]{../imagesCB/partie2-jeu1-fr40-bruit4.jpg} \\
$c_r=3.0$                                                             & $c_r=4.0$ \\
\hline
\end{tabular}
\end{center}

\pagebreak

Ici le bruit $b=0.6*min_{i \neq j}(d(P_i,P_j))$
\begin{center}
\begin{tabular}{|c|c|}
\hline
\includegraphics[width=8cm]{../imagesCB/partie2-jeu1-fr10-bruit6.jpg} & \includegraphics[width=8cm]{../imagesCB/partie2-jeu1-fr11-bruit6.jpg} \\
$c_r=1.0$                                                             & $c_r=1.1$ \\
\hline
\includegraphics[width=8cm]{../imagesCB/partie2-jeu1-fr30-bruit6.jpg} & \includegraphics[width=8cm]{../imagesCB/partie2-jeu1-fr40-bruit6.jpg} \\
$c_r=3.0$                                                             & $c_r=4.0$ \\
\hline
\end{tabular}
\end{center}

\pagebreak

Ici le bruit $b=0.8*min_{i \neq j}(d(P_i,P_j))$
\begin{center}
\begin{tabular}{|c|c|}
\hline
\includegraphics[width=8cm]{../imagesCB/partie2-jeu1-fr10-bruit8.jpg} & \includegraphics[width=8cm]{../imagesCB/partie2-jeu1-fr11-bruit8.jpg} \\
$c_r=1.0$                                                             & $c_r=1.1$ \\
\hline
\includegraphics[width=8cm]{../imagesCB/partie2-jeu1-fr30-bruit8.jpg} & \includegraphics[width=8cm]{../imagesCB/partie2-jeu1-fr40-bruit8.jpg} \\
$c_r=3.0$                                                             & $c_r=4.0$ \\
\hline
\end{tabular}
\end{center}

\pagebreak

Ici le bruit $b=0.2*min_{i \neq j}(d(P_i,P_j))$
\begin{center}
\begin{tabular}{|c|c|}
\hline
\includegraphics[width=8cm]{../imagesCB/partie2-jeu2-fr10-bruit2.jpg} & \includegraphics[width=8cm]{../imagesCB/partie2-jeu2-fr11-bruit2.jpg} \\
$c_r=1.0$                                                             & $c_r=1.1$ \\
\hline
\includegraphics[width=8cm]{../imagesCB/partie2-jeu2-fr12-bruit2.jpg} & \includegraphics[width=8cm]{../imagesCB/partie2-jeu2-fr40-bruit2.jpg} \\
$c_r=1.2$                                                             & $c_r=4.0$ \\
\hline
\end{tabular}
\end{center}

\pagebreak

Ici le bruit $b=0.4*min_{i \neq j}(d(P_i,P_j))$
\begin{center}
\begin{tabular}{|c|c|}
\hline
\includegraphics[width=8cm]{../imagesCB/partie2-jeu2-fr10-bruit4.jpg} & \includegraphics[width=8cm]{../imagesCB/partie2-jeu2-fr11-bruit4.jpg} \\
$c_r=1.0$                                                             & $c_r=1.1$ \\
\hline
\includegraphics[width=8cm]{../imagesCB/partie2-jeu2-fr12-bruit4.jpg} & \includegraphics[width=8cm]{../imagesCB/partie2-jeu2-fr40-bruit4.jpg} \\
$c_r=1.2$                                                             & $c_r=4.0$ \\
\hline
\end{tabular}
\end{center}

\pagebreak

Ici le bruit $b=0.6*min_{i \neq j}(d(P_i,P_j))$
\begin{center}
\begin{tabular}{|c|c|}
\hline
\includegraphics[width=8cm]{../imagesCB/partie2-jeu2-fr10-bruit6.jpg} & \includegraphics[width=8cm]{../imagesCB/partie2-jeu2-fr11-bruit6.jpg} \\
$c_r=1.0$                                                             & $c_r=1.1$ \\
\hline
\includegraphics[width=8cm]{../imagesCB/partie2-jeu2-fr12-bruit6.jpg} & \includegraphics[width=8cm]{../imagesCB/partie2-jeu2-fr40-bruit6.jpg} \\
$c_r=1.2$                                                             & $c_r=4.0$ \\
\hline
\end{tabular}
\end{center}

\pagebreak

Ici le bruit $b=0.8*min_{i \neq j}(d(P_i,P_j))$
\begin{center}
\begin{tabular}{|c|c|}
\hline
\includegraphics[width=8cm]{../imagesCB/partie2-jeu2-fr10-bruit8.jpg} & \includegraphics[width=8cm]{../imagesCB/partie2-jeu2-fr11-bruit8.jpg} \\
$c_r=1.0$                                                             & $c_r=1.1$ \\
\hline
\includegraphics[width=8cm]{../imagesCB/partie2-jeu2-fr12-bruit8.jpg} & \includegraphics[width=8cm]{../imagesCB/partie2-jeu2-fr40-bruit8.jpg} \\
$c_r=1.2$                                                             & $c_r=4.0$ \\
\hline
\end{tabular}
\end{center}

\paragraph{Observations :}
On voit que lorsque le points sont équirépartis, on peu prendre un $r$ proche du pas de l'échantillonage. Lorsque les points sont irrégulièrement
répartis, il convient de prendre un $r$ grand pour avoir des résultats corrects.

%% Lorsque le bruit reste raisonnable (en dessous de $0.8*min_{i \neq j}(d(P_i,P_j))$ dans ces exemples), on observe que plus le bruit est grand,
%% plus il faut que le rayon $r$ soit grand pour avoir un résultat cohérent. On voit aussi que cela ne sert à rien de trop augmenter $r$.
%% Lorsque le bruit devient trop fort, les résultats sont quasiment les mêmes quelque soit $r$.

\pagebreak

\section{Partie surfaces}
\subsection{Structure générale du programme}
En plus des fichiers sources fournis, le programme est aussi composé des fichiers surface.cpp et surface.hpp. Ces deux fichiers
sont similaires aux fichiers courbe.cpp et courbe.hpp. surface.cpp et surface.hpp contiennent les mêmes fonctions et structures que
courbe.cpp et courbe.hpp sauf qu'il y a une dimension en plus. La structure {\tt grille} est donc composée d'un tableau à 3 dimensions,
la structure {\tt Point3} est remplacée par la strucutre {\tt Point4} etc... L'équivalent de la méthode {\tt calc\_iso\_courbe} est la méthode
{\tt calc\_iso\_surf}. La principale différence entre la partie surfaces et la partie courbe est dans la fonction {\tt calc\_iso\_surf}.

Cette méthode calcule la surface iso-valeur à partir de la grille régulière des distances signées. L'idée de cette méthode est simple, on parcourt
toutes les mailles de la grille. Chaque maille est décomposée en 6 tétrèdres. Les points des tétraèdres sont en dimension 4. La quatrième dimension
étant la valeur de la grille en $(i,j,k)$, c'est à dire la distance signée. L'intersection de chaque tétrèdre avec l'hyperplan $w=0$ est calculée si elle existe.
A la suite de cette méthode, nous avons une liste de triangles définissant la surface iso-valeur.

Certains jeux de données étant très volumineux, j'ai utilisé la distance au carré lorsqu'il s'agissait de comparer des distances. Cela évite de calculer une
racine carée ce qui est très coûteux.

\subsection{Manuel utilisateur}
L'utilisateur est entièrement guidé par le programme. L'iso-surface est écrite dans le fichier isosurf.list, et lorsqu'il s'agit
d'un jeu de données sans normales orientées, les normales orientées calculées par le programme sont écrites dans le fichier normales.list.
Si l'utilisateur lance le programme sur un jeu de données sans normales orientées, il lui sera demandé le facteur du bruit $b=c_b*g$, donc
$c_b$ où $g$ est le pas de l'échantillonage. Il lui sera aussi demandé le facteur multiplicatif du rayon $r=c_r*g$ impliqué dans le calcul des normales,
c'est à dire $c_r$.

\pagebreak

\subsection{Résultats}
\subsubsection{Avec les normales orientées}
Voici les résultats pour tous les jeux de test avec les normales orientées.

\begin{center}
\begin{tabular}{|c|c|}
\hline
\includegraphics[width=8cm]{../imagesSurf/donnees1.png} & \includegraphics[width=8cm]{../imagesSurf/donnees2.png}\\
\hline
\end{tabular}
\begin{tabular}{|c|}
\includegraphics[width=8cm]{../imagesSurf/donnees3.png}\\
\hline
\end{tabular}
\end{center}

\pagebreak

\subsubsection{Sans les normales orientées}
\begin{center}
\begin{tabular}{|c|c|}
\hline
\includegraphics[width=8cm]{../imagesSurf/selle-fr11.png} & \includegraphics[width=8cm]{../imagesSurf/cactus11.png}\\
\hline
\includegraphics[width=8cm]{../imagesSurf/cat-fr11.png}   & \includegraphics[width=8cm]{../imagesSurf/mannequin-fr11.png}\\
\hline
\end{tabular}
\begin{tabular}{|c|}
\hline
\includegraphics[width=8cm]{../imagesSurf/club-fr11.png}\\
\hline
\end{tabular}
\end{center}

\pagebreak

\subsubsection{Variation du rayon $r$}
On fait varier ici $r=c_r*g$ en fonction de $c_r$ impliqué dans le calcul des normales où $g$ est le grain d'échantillonnage.

\begin{center}
\begin{tabular}{|c|c|}
\hline
\includegraphics[width=8cm]{../imagesSurf/mannequin-fr10.png} & \includegraphics[width=8cm]{../imagesSurf/mannequin2-fr11.png}\\
$c_r=1$                                                       & $c_r=1.1$\\
\hline
\includegraphics[width=8cm]{../imagesSurf/mannequin-fr12.png} & \includegraphics[width=8cm]{../imagesSurf/mannequin-fr13.png}\\
$c_r=1.2$                                                     & $c_r=1.3$\\
\hline
\end{tabular}
\begin{tabular}{|c|c|}
\hline
\includegraphics[width=8cm]{../imagesSurf/mannequin-fr15.png} & \includegraphics[width=8cm]{../imagesSurf/mannequin-fr50.png}\\
$c_r=1.5$  & $c_r=5.0$\\
\hline
\end{tabular}
\end{center}

\paragraph{Observations :}On voit qu'un rayon trop grand donne de mauvais résultats dans les endroits où la surface a
beaucoup de variations.

\pagebreak

\subsubsection{Variation du bruit}
On fait varier ici le bruit b $b=c_b*g$ en fonction de $c_b$ impliqué dans le calcul des normales où $g$ est le grain d'échantillonnage.

\begin{center}
\begin{tabular}{|c|c|}
\hline
\includegraphics[width=8cm]{../imagesBS/selle-br005-fr11.png} & \includegraphics[width=8cm]{../imagesBS/selle-br008-fr11.png}\\
$c_b=0.05$                                                       & $c_b=0.08$\\
\hline
\includegraphics[width=8cm]{../imagesBS/selle-br01-fr11.png} & \includegraphics[width=8cm]{../imagesBS/selle-br02-fr11.png}\\
$c_b=0.1$                                                     & $c_b=0.2$\\
\hline
\end{tabular}
\end{center}


 \end{document}
