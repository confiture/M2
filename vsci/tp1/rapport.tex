\documentclass[11pt,a4paper]{article}

\usepackage{pdfpages}
\usepackage{amsfonts}
\usepackage{amsmath}
\usepackage{amssymb}
\usepackage{amsthm}
\usepackage[francais]{babel}
\usepackage[dvips,lmargin=3cm,rmargin=3cm,tmargin=3cm,bmargin=4cm]{geometry}
\usepackage{graphicx}
\usepackage{hyperref}
\usepackage[utf8]{inputenc}
\usepackage{listings}
\usepackage{textcomp}
\usepackage{xcolor}
\usepackage{float}
\usepackage[T1]{fontenc}
%\usepackage[nottoc, notlof, notlot]{tocbibind}
\usepackage{graphicx}

% Macros pratique pour les maths.
\newcommand{\bs}{\symbol{92}}
\newcommand{\Z}{\mathbb{Z}}
\newcommand{\Q}{\mathbb{Q}}
\newcommand{\N}{\mathbb{N}}
\newcommand{\R}{\mathbb{R}}
\newcommand{\C}{\mathbb{C}}
\newcommand{\K}{\mathbb{K}}
\newcommand{\B}{\mathcal{B}}
\newcommand{\tab}{\hspace*{1cm}}
\newcommand{\norm}[1]{\left\vert \left\vert {#1}  \right\vert \right\vert}
\newcommand{\abs}[1]{ \left\vert {#1}  \right\vert}
\newcommand{\vect}[1]{\overrightarrow{#1}}
\newcommand{\scal}[2]{\langle #1 \vert #2 \rangle}
\newcommand{\exems}{$ $ \\ \noindent{{\textbf{Exemples.~}}}}
\newcommand{\exem}{$ $ \\ \noindent{{\textbf{Exemple.~}}}}
\newcommand{\rem}{$ $\\\ \noindent{{\textbf{Remarque.~}}}}
\newcommand{\rems}{$ $ \\ \noindent{{\textbf{Remarques.~}}}}
\newcommand{\rap}{$ $ \\ \noindent{{\textbf{Rappel.~}}}}
\newcommand{\car}[1]{|#1|}
\newcommand{\card}{\mbox{card}}
\newcommand{\stab}{\mbox{Stab}}
\newcommand{\ba}[1]{\overline{#1}}
\newcommand{\inte}[1]{ [\![{#1} ]\!]}

% Configuration du package lstlisting pour du C++.
\lstset{
  basicstyle=\small,
  breaklines=true,
  commentstyle=\color[rgb]{0.3,0.3,0.3}\textit,
  emphstyle=\rmfamily\color{blue},
  frame=single,
  keywordstyle=\color{blue},
  language=C++,
  lineskip=0pt,
  numbers=left,
  numbersep=5pt,
  numberstyle=\tt\small\color{gray},
  showstringspaces=false,
  tabsize=4,
  texcl=true
}

\title{
  \huge{\bf TP1 - Calcul de courbes iso-valeur sur une triangulation}
}

\author{
    \textsc{Koraa} Galel \\
    \textsc{Trlin} Moreno
}

\date{7 janvier 2011}

\begin{document}
 \maketitle
 %\tableofcontents
 \pagebreak
%%%%%%%%%%%%%%%%%%%%%%%%%%%%%%

  \section{Le code}
  On a implémenté tout d'abord une fonction intersection entre un segment et un plan :\\

  \begin{lstlisting}[language=c++]
/**
 *Retourne l'intersection du segment [p1,p2] avec le plan z=v.
 *
 */
Point3 intersection(const Point3& p1,const Point3& p2,int v){
	if(v==p1.z)return p1;
	if(v==p2.z)return p2;

	double a = (v-p2.z)/(p1.z-p2.z);
	return a*p1+(1-a)*p2;
}
  \end{lstlisting}

  Nous avons ensuite complété la fonction {\tt calcul\_isocourbe} :\\

  \begin{lstlisting}[language=c++]
	N=0;
	bool e2; // booléen qui indique si on est sur le deuxième sommet du segment (E2[N])
	for(int iT=0;iT<nT;iT++){ // boucle sur les mailles
		e2=false;
		for(int ar=0;ar<3;ar++){ //boucle sur les arêtes
			if(S[T[iT][ar]].z==v && S[T[iT][(ar+1)%3]].z==v){ // cas où l'arête est sur le plan
				E1[N]=S[T[iT][ar]];
				E2[N]=S[T[iT][(ar+1)%3]];
				N++;
				break;
			}
			else if((S[T[iT][ar]].z-v)*(S[T[iT][(ar+1)%3]].z-v)<=0){ // cas où le plan intersecte l'arête en un seul point
				if(e2){
					E2[N]=intersection(S[T[iT][ar]],S[T[iT][(ar+1)%3]],v);
					N++;
					break;
				}
				else{
					E1[N]=intersection(S[T[iT][ar]],S[T[iT][(ar+1)%3]],v);
					e2=true;
				}
			}
		}
	}
  \end{lstlisting}

  Notre programme {\tt tp1} prend en paramètre la hauteur,
  le fichier d'entrée, le fichier de sortie et la couleur (r,g,b) de
  l'iso-courbe.

  Pour obtenir un grand nombre d'iso-courbes avec un dégradé de couleur,
  nous avons écrit un script shell qui fait ceci. Ce script prend
  en entrée le fichier à traiter, la hauteur minimale, le pas des
  iso-courbes et la hauteur maximale.
  Voici le shell :\\

  \begin{lstlisting}[language=bash]
#!/bin/sh

ext=.vect
for i in $( seq $2 $3 $4 ) # boucle sur les hauteurs des iso-courbes
do
	r=$(echo "($i - $2)/($4 - $2)" | bc -l) # la couleur rouge
	b=$(echo "1 - $r" | bc -l)              # la couleur bleue
	bin/tp1 $i $1 $5$i$ext $r 0.0 $b        # exécution du programme
done

  \end{lstlisting}


  \section{Algorithmes}
  \subsection{Composantes connexes}

  \subsection{Axe médian}

  \subsection{Opérateurs morphologiques}


  \section{Exercice 1}

  \pagebreak
  \section{Exercice 2}


%   \begin{center}
%
%   Voici le résultat de l'interpolation pour $(\Delta i, \Delta j)= (7.1,-3.8) :$ \\ ~\\
%   \begin{tabular}{cc}
%    \includegraphics[height=6cm]{bateauT1/bateauT.png} & \includegraphics[height=6cm]{bateauT1/ErrbateauT.png} \\
%    {\it Image interpolée} & {\it Différence entre l'image interpolée et l'image originale}
%   \end{tabular}
%   \end{center}
%
% \pagebreak
%
%   \begin{center}
%    \includegraphics[height=7cm]{houseT_-4.7_3.8/house.png}
%    {\it ~\\ Image originale \\ ~\\}
%
%   Voici le résultat de l'interpolation pour $(\Delta i, \Delta j)= (-4.7, 3.8) :$ \\ ~\\
%   \begin{tabular}{cc}
%    \includegraphics[height=7cm]{houseT_-4.7_3.8/houseT.png} & \includegraphics[height=7cm]{houseT_-4.7_3.8/ErrHouseT.png} \\
%    {\it Image interpolée} & {\it Différence entre l'image interpolée et l'image originale}
%   \end{tabular}
%   \end{center}
%
% \paragraph{Observations :}
% On voit se former des bandes noires sur les bords dues à la translation, de plus
% l'interpolation floute l'image.
%
% \pagebreak
%
% \section{L'objet tracké}
% Voici quelques résultats de la traque de l'objet :
% \begin{center}
%  \begin{tabular}{|c|c|c|c|}
%  \hline
%  taz001 & taz002 & taz003 & taz004 \\
%  \includegraphics[height=3cm]{sauveCadres/taz001.png} & \includegraphics[height=3cm]{sauveCadres/taz002.png} &
%  \includegraphics[height=3cm]{sauveCadres/taz003.png} & \includegraphics[height=3cm]{sauveCadres/taz004.png}\\
%  \hline
%  taz005 & taz006 & taz007 & taz008 \\
%  \includegraphics[height=3cm]{sauveCadres/taz005.png} & \includegraphics[height=3cm]{sauveCadres/taz006.png} &
%  \includegraphics[height=3cm]{sauveCadres/taz007.png} & \includegraphics[height=3cm]{sauveCadres/taz008.png}\\
%  \hline
%  \end{tabular}
% \end{center}

\end{document}
