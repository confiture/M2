\documentclass[11pt,a4paper]{article}

\usepackage{pdfpages}
\usepackage{amsfonts}
\usepackage{amsmath}
\usepackage{amssymb}
\usepackage{amsthm}
\usepackage[francais]{babel}
\usepackage[dvips,lmargin=3cm,rmargin=3cm,tmargin=3cm,bmargin=4cm]{geometry}
\usepackage{graphicx}
\usepackage{hyperref}
\usepackage[utf8]{inputenc}
\usepackage{listings}
\usepackage{textcomp}
\usepackage{xcolor}
\usepackage{float}
\usepackage[T1]{fontenc}
%\usepackage[nottoc, notlof, notlot]{tocbibind}
\usepackage{graphicx}

\usepackage{subfig}

% Macros pratique pour les maths.
\newcommand{\bs}{\symbol{92}}
\newcommand{\Z}{\mathbb{Z}}
\newcommand{\Q}{\mathbb{Q}}
\newcommand{\N}{\mathbb{N}}
\newcommand{\R}{\mathbb{R}}
\newcommand{\C}{\mathbb{C}}
\newcommand{\K}{\mathbb{K}}
\newcommand{\B}{\mathcal{B}}
\newcommand{\tab}{\hspace*{1cm}}
\newcommand{\norm}[1]{\left\vert \left\vert {#1}  \right\vert \right\vert}
\newcommand{\abs}[1]{ \left\vert {#1}  \right\vert}
\newcommand{\vect}[1]{\overrightarrow{#1}}
\newcommand{\scal}[2]{\langle #1 \vert #2 \rangle}
\newcommand{\exems}{$ $ \\ \noindent{{\textbf{Exemples.~}}}}
\newcommand{\exem}{$ $ \\ \noindent{{\textbf{Exemple.~}}}}
\newcommand{\rem}{$ $\\\ \noindent{{\textbf{Remarque.~}}}}
\newcommand{\rems}{$ $ \\ \noindent{{\textbf{Remarques.~}}}}
\newcommand{\rap}{$ $ \\ \noindent{{\textbf{Rappel.~}}}}
\newcommand{\car}[1]{|#1|}
\newcommand{\card}{\mbox{card}}
\newcommand{\stab}{\mbox{Stab}}
\newcommand{\ba}[1]{\overline{#1}}
\newcommand{\inte}[1]{ [\![{#1} ]\!]}

% Configuration du package lstlisting pour du C++.
% \lstset{
%   basicstyle=\small,
%   breaklines=true,
%   commentstyle=\color[rgb]{0.3,0.3,0.3}\textit,
%   emphstyle=\rmfamily\color{blue},
%   frame=single,
%   keywordstyle=\color{blue},
%   language=C++,
%   lineskip=0pt,
%   numbers=left,
%   numbersep=5pt,
%   numberstyle=\tt\small\color{gray},
%   showstringspaces=false,
%   tabsize=4,
%   texcl=true
% }
\definecolor{colKeys}{rgb}{0,0,1}
\definecolor{colIdentifier}{rgb}{0,0,0}
\definecolor{colComments}{rgb}{0,0.5,1}
\definecolor{colString}{rgb}{0.6,0.1,0.1}

\lstset{%configuration de listings
float=hbp, %
basicstyle=\ttfamily \small, %
identifierstyle=\color{colIdentifier}, %
keywordstyle=\color{colKeys}, %
stringstyle=\color{colString}, %
commentstyle=\color{colComments}, %
columns=flexible, %
tabsize=4, %
frame=trBL, %
frameround=tttt, %
extendedchars=true, %
showspaces=false, %
showstringspaces=false, %
numbers=left, %
numberstyle=\tiny, %
breaklines=true, %
breakautoindent=true, %
captionpos=b, %
xrightmargin=0cm, %
xleftmargin=0cm,
language=scilab,
commentstyle=\textit
}



\title{
  \huge{\bf TP5 - Simplification de lignes polygonales}
}

\author{
    \textsc{Koraa} Galel \\
    \textsc{Trlin} Moreno \\
}

\date{18 février 2011}

\begin{document}

 \maketitle
  \begin{center}
   \includegraphics[height=4cm]{../images/imag.png}

  \end{center}
 \tableofcontents

 \pagebreak
%%%%%%%%%%%%%%%%%%%%%%%%%%%%%%

  \section{Objectif}
  Le but de ce TP est de comparer l'algorithme de décimation et l'algorithme de Douglas-Peuker pour la simplification de lignes polygonales à l'aide de différents exemples.
  \section{Simplification par décimation}

\subsection{Même jeu de données et un même taux de compression}

\subsubsection{arbre.data}

\begin{center}
 \begin{tabular}{|c|c|}
\hline
 \includegraphics[width=8cm]{../images/arbre_init.png} & \includegraphics[width=8cm]{../images/arbre_simpl_1_280.png} \\
image initiale   &  déviation et taux de compression = 80 \\
\hline
 \includegraphics[width=8cm]{../images/arbre_simpl_2_280.png} & \includegraphics[width=8cm]{../images/arbre_simpl_3_280.png} \\
longueur et taux de compression = 80   &  angle et taux de compression = 80   \\
\hline
 \end{tabular}
 \end{center}
\subsubsection{hippocampe.data}
\begin{center}
 \begin{tabular}{|c|c|}
\hline
 \includegraphics[height=10cm]{../images/hippocampe.png} & \includegraphics[height=10cm]{../images/hippo_simp_1_235.png} \\
image initiale   &  déviation et taux de compression = 35 \\
\hline
 \includegraphics[height=10cm]{../images/hippo_simp_2_235.png} & \includegraphics[height=10cm]{../images/hippo_simp_3_235.png} \\
longueur et taux de compression = 35  &  angle et taux de compression =35  \\
\hline
 \end{tabular}
 \end{center}

\paragraph{Observations :}
On peut observer que les résultats obtenus avec la fonction critère déviation sont les meilleurs. Ceux obtenus avec la fonction
critère angle sont les moins bons.


\subsection{Même jeu de données et une même fonction-critère}

\subsubsection{fleur.data}
Fonction critère : déviation.
\begin{center}
 \begin{tabular}{|c|c|}
\hline
 \includegraphics[width=8cm]{../images/fleur.png} & \includegraphics[width=8cm]{../images/fleur_simp_1_225.png} \\
image initiale   &  déviation et taux de compression = 25 \\
\hline
 \includegraphics[width=8cm]{../images/fleur_simp_1_250} & \includegraphics[width=8cm]{../images/fleur_simp_1_275.png} \\
déviation et taux de compression = 50   &  déviation et taux de compression = 75   \\
\hline
 \end{tabular}
 \end{center}
\pagebreak

Fonction critère : longueur.
\begin{center}
 \begin{tabular}{|c|c|}
\hline
 \includegraphics[width=8cm]{../images/fleur.png} & \includegraphics[width=8cm]{../images/fleur_simp_2_225.png} \\
image initiale   &  longueur et taux de compression = 25 \\
\hline
 \includegraphics[width=8cm]{../images/fleur_simp_2_250} & \includegraphics[width=8cm]{../images/fleur_simp_2_275.png} \\
longueur et taux de compression = 50   &  longueur et taux de compression = 75   \\
\hline
 \end{tabular}
 \end{center}
\pagebreak

Fonction critère : angle.
\begin{center}
 \begin{tabular}{|c|c|}
\hline
 \includegraphics[width=8cm]{../images/fleur.png} & \includegraphics[width=8cm]{../images/fleur_simp_3_225.png} \\
image initiale   &  angle et taux de compression = 25 \\
\hline
 \includegraphics[width=8cm]{../images/fleur_simp_3_250} & \includegraphics[width=8cm]{../images/fleur_simp_3_275.png} \\
angle et taux de compression = 50   &  angle et taux de compression = 75   \\
\hline
 \end{tabular}
 \end{center}
\pagebreak

\subsubsection{france.data}

Fonction critère : déviation.
\begin{center}
 \begin{tabular}{|c|c|}
\hline
 \includegraphics[width=8cm]{../images/france.png} & \includegraphics[width=8cm]{../images/france_simp_1_225.png} \\
image initiale   &  déviation et taux de compression = 25 \\
\hline
 \includegraphics[width=8cm]{../images/france_simp_1_250} & \includegraphics[width=8cm]{../images/france_simp_1_275} \\
déviation et taux de compression = 50   & déviation et taux de compression = 75   \\
\hline
 \end{tabular}
 \end{center}


\pagebreak
Fonction critère : longueur.
\begin{center}
 \begin{tabular}{|c|c|}
\hline
 \includegraphics[width=8cm]{../images/france.png} & \includegraphics[width=8cm]{../images/france_simp_2_225.png} \\
image initiale   &  longueur et taux de compression = 25 \\
\hline
 \includegraphics[width=8cm]{../images/france_simp_2_250} & \includegraphics[width=8cm]{../images/france_simp_2_275} \\
longueur et taux de compression = 50   & longueur et taux de compression = 75   \\
\hline
 \end{tabular}
 \end{center}

\pagebreak
Fonction critère : angle.
\begin{center}
 \begin{tabular}{|c|c|}
\hline
 \includegraphics[width=8cm]{../images/france.png} & \includegraphics[width=8cm]{../images/france_simp_3_225.png} \\
image initiale   &  angle et taux de compression = 25 \\
\hline
 \includegraphics[width=8cm]{../images/france_simp_3_250} & \includegraphics[width=8cm]{../images/france_simp_3_275} \\
angle et taux de compression = 50   & angle et taux de compression = 75   \\
\hline
 \end{tabular}
 \end{center}

\paragraph{Observations :}
On remarque que plus le taux de compression est grand et moins il y a de sommets. On perd de l'information sur l'image donc
en qualité.
En effet le taux de compression étant le rapport entre le nombre de sommets du polygone initial et le nombre de sommets du polygone simplifié,
plus celui-ci augmente et plus le nombre de sommets du polygone simplifié diminue.

 \section{Simplification par la méthode de Douglas-Peuker}


\subsection{Algorithme de Douglas-Peuker}
   \subsubsection{Le code}
\begin{lstlisting}[language=c++]
 /**
 *Exécute l'algorithme de Douglas-Parker pour la distance d sur la liste de points lpts.
 *Lorsque la procédure est lancée, deb doit pointer sur le premier élément de la liste,
 *et fin doit pointer sur le dernier élément de la liste et non sur lpts.end().
 */
void douglas(double d,std::list<Point2> & lpts,std::list<Point2>::iterator deb,std::list<Point2>::iterator fin){
    if(deb!=fin){
        double a,b,h,l,s,dj;
        dj=0;
        std::list<Point2>::iterator it=deb;
        std::list<Point2>::iterator ptLoin=deb;

        it++;
        if(it!=fin){
            while(it!=fin){
                a=dist((*deb),(*it));
                b=dist((*fin),(*it));
                l=dist((*deb),(*fin));
                s=(a*a-b*b+l*l)/(2*l);

                if(0<=s && s<=l){
                    h=sqrt(a*a-s*s);
                }
                else if(s<0){
                    h=a;
                }
                else{
                    h=b;
                }

                if(h>dj){
                    dj=h;
                    ptLoin=it;
                }

                it++;
            }

            if(dj<d){
                it=deb;
                it++;
                lpts.erase(it,fin);
            }
            else{
                douglas(d,lpts,deb,ptLoin);
                douglas(d,lpts,ptLoin,fin);
            }
        }
    }
}
\end{lstlisting}


\subsection{Tests}
\subsubsection{demicercle.data}

\begin{center}
 \begin{tabular}{|c|c|}
\hline
 \includegraphics[width=8cm]{../images/demicercle.png} & \includegraphics[width=8cm]{../imagesDoug/demicercle1.png} \\
image initiale   &  d=1, taux de compression = 1.53846\\
\hline
 \includegraphics[width=8cm]{../imagesDoug/demicercle2.png} & \includegraphics[width=8cm]{../imagesDoug/demicercle5.png} \\
d=2, taux de compression = 2.77778&  d=5, taux de compression = 5 \\
\hline
 \includegraphics[width=8cm]{../imagesDoug/demicercle10.png} & \includegraphics[width=8cm]{../imagesDoug/demicercle20.png} \\
d=10, taux de compression = 5.88235&  d=20, taux de compression = 8.33333\\
\hline
 \includegraphics[width=8cm]{../imagesDoug/demicercle50.png} & \includegraphics[width=8cm]{../imagesDoug/demicercle100.png} \\
d=50, taux de compression = 11.1111   &  d=100, taux de compression = 20 \\
\hline
 \end{tabular}
 \end{center}

\begin{center}
  \begin{tabular}{|c|}
  \hline
 \includegraphics[width=8cm]{../imagesDoug/demicercle200.png} \\
d=200, taux de compression = 20\\
\hline
 \end{tabular}
\end{center}

\subsubsection{France.data}

\begin{center}
 \begin{tabular}{|c|c|}
\hline
 \includegraphics[width=8cm]{../images/france.png} & \includegraphics[width=8cm]{../imagesDoug/France1.png} \\
image initiale   &  d=1, taux de compression = 11.4682  \\
\hline
 \includegraphics[width=8cm]{../imagesDoug/France2.png} & \includegraphics[width=8cm]{../imagesDoug/France5.png} \\
d=2, taux de compression = 18.2564    &  d=5, taux de compression = 34.7255  \\
\hline
 \end{tabular}
 \end{center}

\begin{center}
 \begin{tabular}{|c|c|}
\hline
 \includegraphics[width=8cm]{../imagesDoug/France10.png} & \includegraphics[width=8cm]{../imagesDoug/France20.png} \\
d=10, taux de compression = 62.8576  &  d=20, taux de compression = 127.327 \\
\hline
 \includegraphics[width=8cm]{../imagesDoug/France50.png} & \includegraphics[width=8cm]{../imagesDoug/France100.png} \\
d=50, taux de compression = 348.474   &  d=100, taux de compression = 945.857  \\
\hline
 \end{tabular}
 \end{center}

\begin{center}
 \begin{tabular}{|c|}
  \hline
 \includegraphics[width=8cm]{../imagesDoug/France200.png} \\
d=200, taux de compression = 1655.25\\
\hline
 \end{tabular}
\end{center}


\subsubsection{fleur.data}

\begin{center}
 \begin{tabular}{|c|c|}
\hline
 \includegraphics[width=8cm]{../images/fleur.png} & \includegraphics[width=8cm]{../imagesDoug/fleur1.png} \\
image initiale   &  d=1, taux de compression = 3.62319  \\
\hline
 \includegraphics[width=8cm]{../imagesDoug/fleur2.png} & \includegraphics[width=8cm]{../imagesDoug/fleur5.png} \\
d=2, taux de compression = 5.61798    &  d=5, taux de compression = 9.17431  \\
\hline
 \end{tabular}
 \end{center}

\begin{center}
 \begin{tabular}{|c|c|}
\hline
 \includegraphics[width=8cm]{../imagesDoug/fleur10.png} & \includegraphics[width=8cm]{../imagesDoug/fleur20.png} \\
d=10, taux de compression = 13.3333  &  d=20, taux de compression = 17.5439 \\
\hline
 \includegraphics[width=8cm]{../imagesDoug/fleur50.png} & \includegraphics[width=8cm]{../imagesDoug/fleur100.png} \\
d=50, taux de compression = 29.4118   &  d=100, taux de compression = 40  \\
\hline
 \end{tabular}
 \end{center}

\begin{center}
 \begin{tabular}{|c|}
  \hline
 \includegraphics[width=8cm]{../imagesDoug/fleur200.png} \\
d=200, taux de compression = 66.6667\\
\hline
 \end{tabular}
\end{center}


\subsubsection{arbre.data}

\begin{center}
 \begin{tabular}{|c|c|}
\hline
 \includegraphics[width=8cm]{../images/arbre_init.png} & \includegraphics[width=8cm]{../imagesDoug/arbre1.png} \\
image initiale   &  d=1, taux de compression = 7.7723  \\
\hline
 \includegraphics[width=8cm]{../imagesDoug/arbre2.png} & \includegraphics[width=8cm]{../imagesDoug/arbre5.png} \\
d=2, taux de compression = 14.6547    &  d=5, taux de compression = 32.9658  \\
\hline
 \end{tabular}
 \end{center}

\begin{center}
 \begin{tabular}{|c|c|}
\hline
 \includegraphics[width=8cm]{../imagesDoug/arbre10.png} & \includegraphics[width=8cm]{../imagesDoug/arbre20.png} \\
d=10, taux de compression = 50.7245 &  d=20, taux de compression = 80.3137\\
\hline
 \includegraphics[width=8cm]{../imagesDoug/arbre50.png} & \includegraphics[width=8cm]{../imagesDoug/arbre100.png} \\
d=50, taux de compression = 121.363   &  d=100, taux de compression = 190.512  \\
\hline
 \end{tabular}
 \end{center}

\begin{center}
 \begin{tabular}{|c|}
  \hline
 \includegraphics[width=8cm]{../imagesDoug/arbre200.png} \\
d=200, taux de compression = 455.111\\
\hline
 \end{tabular}
\end{center}
\paragraph{Observations :}
On remarque que plus la distance-seuil augmente et moins il y a de sommets. On perd de l'information sur l'image donc
en qualité.

\subsection{Comparaison des algorithmes de décimation et de Douglas-Peuker}
\subsubsection{France.data}


\begin{center}
 \begin{tabular}{|c|c|}
\hline
algorithme de décimation & algorithme de Douglas-Peuker \\
\hline
 \includegraphics[width=8cm]{../images/france_decim1.png} & \includegraphics[width=8cm]{../imagesDoug/France1.png} \\
déviation, taux de compression = 11.4682   &  d=1, taux de compression = 11.4682  \\
\hline
 \includegraphics[width=8cm]{../images/france_decim2.png} & \includegraphics[width=8cm]{../imagesDoug/France2.png} \\
déviation, taux de compression = 18.2564   & d=2, taux de compression = 18.2564  \\
\hline
 \end{tabular}
 \end{center}

\begin{center}
 \begin{tabular}{|c|c|}
\hline
algorithme de décimation & algorithme de Douglas-Peuker \\
\hline
 \includegraphics[width=8cm]{../images/france_decim5.png} & \includegraphics[width=8cm]{../imagesDoug/France5.png} \\
déviation, taux de compression = 34.7255   &  d=5, taux de compression = 34.7255  \\
\hline
 \includegraphics[width=8cm]{../images/france_decim10.png} & \includegraphics[width=8cm]{../imagesDoug/France10.png} \\
déviation, taux de compression = 62.8576 & d=10, taux de compression = 62.8576 \\
\hline
 \end{tabular}
 \end{center}

\begin{center}
 \begin{tabular}{|c|c|}
\hline
algorithme de décimation & algorithme de Douglas-Peuker \\
\hline
 \includegraphics[width=8cm]{../images/france_decim20.png} & \includegraphics[width=8cm]{../imagesDoug/France20.png} \\
déviation, taux de compression = 127.327  &  d=20, taux de compression = 127.327  \\
\hline
 \includegraphics[width=8cm]{../images/france_decim50.png} & \includegraphics[width=8cm]{../imagesDoug/France50.png} \\
déviation, taux de compression = 348.474 & d=50, taux de compression = 348.474 \\
\hline
 \end{tabular}
 \end{center}

\begin{center}
 \begin{tabular}{|c|c|}
\hline
algorithme de décimation & algorithme de Douglas-Peuker \\
\hline
 \includegraphics[width=8cm]{../images/france_decim100.png} & \includegraphics[width=8cm]{../imagesDoug/France100.png} \\
déviation, taux de compression = 945.857  &  d=100, taux de compression = 945.857  \\
\hline
 \includegraphics[width=8cm]{../images/france_decim200.png} & \includegraphics[width=8cm]{../imagesDoug/France200.png} \\
déviation, taux de compression = 1655.25 & d=200, taux de compression = 1655.25 \\
\hline
 \end{tabular}
 \end{center}

\paragraph{Observations :}
On remarque des différences entre les deux algorithmes, mais celles-ci ne sont pas notables.


 \end{document}
