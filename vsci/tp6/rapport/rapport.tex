\documentclass[11pt,a4paper]{article}

\usepackage{pdfpages}
\usepackage{amsfonts}
\usepackage{amsmath}
\usepackage{amssymb}
\usepackage{amsthm}
\usepackage[francais]{babel}
\usepackage[dvips,lmargin=3cm,rmargin=3cm,tmargin=3cm,bmargin=4cm]{geometry}
\usepackage{graphicx}
\usepackage{hyperref}
\usepackage[utf8]{inputenc}
\usepackage{listings}
\usepackage{textcomp}
\usepackage{xcolor}
\usepackage{float}
\usepackage[T1]{fontenc}
%\usepackage[nottoc, notlof, notlot]{tocbibind}
\usepackage{graphicx}

\usepackage{subfig}

% Macros pratique pour les maths.
\newcommand{\bs}{\symbol{92}}
\newcommand{\Z}{\mathbb{Z}}
\newcommand{\Q}{\mathbb{Q}}
\newcommand{\N}{\mathbb{N}}
\newcommand{\R}{\mathbb{R}}
\newcommand{\C}{\mathbb{C}}
\newcommand{\K}{\mathbb{K}}
\newcommand{\B}{\mathcal{B}}
\newcommand{\tab}{\hspace*{1cm}}
\newcommand{\norm}[1]{\left\vert \left\vert {#1}  \right\vert \right\vert}
\newcommand{\abs}[1]{ \left\vert {#1}  \right\vert}
\newcommand{\vect}[1]{\overrightarrow{#1}}
\newcommand{\scal}[2]{\langle #1 \vert #2 \rangle}
\newcommand{\exems}{$ $ \\ \noindent{{\textbf{Exemples.~}}}}
\newcommand{\exem}{$ $ \\ \noindent{{\textbf{Exemple.~}}}}
\newcommand{\rem}{$ $\\\ \noindent{{\textbf{Remarque.~}}}}
\newcommand{\rems}{$ $ \\ \noindent{{\textbf{Remarques.~}}}}
\newcommand{\rap}{$ $ \\ \noindent{{\textbf{Rappel.~}}}}
\newcommand{\car}[1]{|#1|}
\newcommand{\card}{\mbox{card}}
\newcommand{\stab}{\mbox{Stab}}
\newcommand{\ba}[1]{\overline{#1}}
\newcommand{\inte}[1]{ [\![{#1} ]\!]}

% Configuration du package lstlisting pour du C++.
% \lstset{
%   basicstyle=\small,
%   breaklines=true,
%   commentstyle=\color[rgb]{0.3,0.3,0.3}\textit,
%   emphstyle=\rmfamily\color{blue},
%   frame=single,
%   keywordstyle=\color{blue},
%   language=C++,
%   lineskip=0pt,
%   numbers=left,
%   numbersep=5pt,
%   numberstyle=\tt\small\color{gray},
%   showstringspaces=false,
%   tabsize=4,
%   texcl=true
% }
\definecolor{colKeys}{rgb}{0,0,1}
\definecolor{colIdentifier}{rgb}{0,0,0}
\definecolor{colComments}{rgb}{0,0.5,1}
\definecolor{colString}{rgb}{0.6,0.1,0.1}

\lstset{%configuration de listings
float=hbp, %
basicstyle=\ttfamily \small, %
identifierstyle=\color{colIdentifier}, %
keywordstyle=\color{colKeys}, %
stringstyle=\color{colString}, %
commentstyle=\color{colComments}, %
columns=flexible, %
tabsize=4, %
frame=trBL, %
frameround=tttt, %
extendedchars=true, %
showspaces=false, %
showstringspaces=false, %
numbers=left, %
numberstyle=\tiny, %
breaklines=true, %
breakautoindent=true, %
captionpos=b, %
xrightmargin=0cm, %
xleftmargin=0cm,
language=scilab,
commentstyle=\textit
}



\title{
  \huge{\bf TP6 - Simplification de triangulation}
}

\author{
    \textsc{Koraa} Galel \\
    \textsc{Trlin} Moreno \\
}

\date{18 février 2011}

\begin{document}

 \maketitle
  \begin{center}
   \includegraphics[height=4cm]{../images/imag.png}

  \end{center}
 \tableofcontents

 \pagebreak
%%%%%%%%%%%%%%%%%%%%%%%%%%%%%%

  \section{Objectif}
Le but de ce TP est d'illustrer le cours par l'utilisation d'un programme de simplification de triangulation basé sur la méthode QEM.

  \section{Simplification sur différents modèles}
\subsection{Horse}

 \begin{tabular}{|c|c|}
\hline
 \includegraphics[width=8cm]{../images/horse.png} & \includegraphics[width=8cm]{../images/horse_t10000.png} \\
image initiale, 39698 triangles   &  10 000 triangles \\
\hline
 \includegraphics[width=8cm]{../images/horse_t1000.png} & \includegraphics[width=8cm]{../images/horse_t250.png} \\
1 000 triangles  &  250 triangles   \\
\hline
 \end{tabular}

\subsection{Mannequin}
\begin{tabular}{|c|c|}
\hline
 \includegraphics[width=8cm]{../images/mannequin.png} & \includegraphics[width=8cm]{../images/mannequin_t800.png} \\
image initiale, 1355 triangles   &  800 triangles \\
\hline
 \includegraphics[width=8cm]{../images/mannequin_t500.png} & \includegraphics[width=8cm]{../images/mannequin_t200.png} \\
500 triangles  &  200 triangles   \\
\hline
\end{tabular}

\subsection{Dragon}
\begin{tabular}{|c|c|}
\hline
 \includegraphics[width=8cm]{../images/dragon.png} & \includegraphics[width=8cm]{../images/dragon_t50000.png} \\
image initiale, 200 000 triangles   &  50 000 triangles \\
\hline
\includegraphics[width=8cm]{../images/dragon_t5000.png} & \includegraphics[width=8cm]{../images/dragon_t1000.png} \\
5 000 triangles  &  1 000 triangles   \\
\hline
\end{tabular}

\paragraph{Observations :}
En divisant le nombre de triangles par 4, on observe quasiment aucune différence entre l'image initiale et celle 
de 50 000 triangles.

\subsection{Bunny}
 \begin{tabular}{|c|c|}
\hline
 \includegraphics[width=8cm]{../images/bunny.png} & \includegraphics[width=8cm]{../images/bunny_t30000.png} \\
image initiale, 69451 triangles   &  30 000 triangles \\
\hline
 \includegraphics[width=8cm]{../images/bunny_t5000.png} & \includegraphics[width=8cm]{../images/bunny_t1000.png} \\
5 000 triangles  &  1 000 triangles   \\
\hline
 \end{tabular}

\subsection{Plane}
\begin{tabular}{|c|c|}
\hline
 \includegraphics[width=8cm]{../images/plane.png} & \includegraphics[width=8cm]{../images/plane_t5000.png} \\
image initiale, 228 317 triangles   &  5 000 triangles \\
\hline
\includegraphics[width=8cm]{../images/plane_t1000.png} & \includegraphics[width=8cm]{../images/plane_t250.png} \\
1 000 triangles  &  250 triangles   \\
\hline
\end{tabular}

\subsection{Happy}
\begin{tabular}{|c|c|}
\hline
 \includegraphics[width=8cm]{../images/happy.png} & \includegraphics[width=8cm]{../images/happy_t5000.png} \\
image initiale, 228 317 triangles   &  5 000 triangles \\
\hline
\includegraphics[width=8cm]{../images/happy_t1000.png} & \includegraphics[width=8cm]{../images/happy_t250.png} \\
1 000 triangles  &  250 triangles   \\
\hline
\end{tabular}

\subsection{Hypersheet}
\begin{tabular}{|c|c|}
\hline
 \includegraphics[width=8cm]{../images/hypersheet2.png} & \includegraphics[width=8cm]{../images/hypersheet_t400.png} \\
image initiale, 897 triangles   &  400 triangles \\
\hline
\includegraphics[width=8cm]{../images/hypersheet_t100.png} & \includegraphics[width=8cm]{../images/hypersheet_t50.png} \\
100 triangles  &  50 triangles   \\
\hline
\end{tabular}

\paragraph{Observations :}
On voit que lorsqu'on simplifie trop le maillage, la topologie de celui-ci peut être modifiée, plus particulièrement,
on observe sur cet exemple (50 triangles) que la simplification du maillage à tendance à boucher les trous.


\section{Simplification du modèle hypersheet.smf avec différentes options}

\begin{center}
\begin{tabular}{|l||c|c|}
\hline
Image initiale, 897 triangles & \includegraphics[width=5.5cm]{../images/hypersheet.png} & \includegraphics[width=5.5cm]{../images/hypersheet_vue2.png} \\
\hline
\begin{tabular}{l}
Simplification avec\\
préservation des\\
bords, 200 triangles
\end{tabular} & \includegraphics[width=5.5cm]{../images/hypersheet_defaut_t200.png}  & \includegraphics[width=5.5cm]{../images/hypersheet_defaut_t200_vue2.png} \\
\hline
\begin{tabular}{l}
Simplification sans\\
préservation des\\
bords, 200 triangles
\end{tabular} & \includegraphics[width=5.5cm]{../images/hypersheet_option_t200.png}  & \includegraphics[width=5.5cm]{../images/hypersheet_option_t200_vue2.png} \\
\hline
\end{tabular}
\end{center}

\paragraph{Observations :}
Ces images illustrent très bien la différence entre les simplifications avec ou sans préservation des bords.
La simplification avec préservation des bords conserve bien mieux le maillage.


\section{Simplification du modèle mannequin.smf avec différentes options}

\begin{center}
\begin{tabular}{|l||c|c|}
\hline
Image initiale, 1350 triangles & \includegraphics[width=5.5cm]{../images/mannequin.png} & \includegraphics[width=5.5cm]{../images/mannequin_vue2.png} \\
\hline
\begin{tabular}{l}
Simplification avec\\
placement optimal \\
des sommets, 400 triangles
\end{tabular} & \includegraphics[width=5.5cm]{../images/mannequin_defaut_t400.png}  & \includegraphics[width=5.5cm]{../images/mannequin_defaut_t400_vue2.png} \\
\hline
\begin{tabular}{l}
Simplification sans\\
placement optimal \\
des sommets, 400 triangles
\end{tabular} & \includegraphics[width=5.5cm]{../images/mannequin_option_t400.png}  & \includegraphics[width=5.5cm]{../images/mannequin_option_t400_vue2.png} \\
\hline
\end{tabular}
\end{center}

\paragraph{Observations :}
On voit que la simplification avec placement optimal des sommets conserve bien mieux le maillage.

\section{Simplification d'images}

\subsection{Image1}
\begin{center}
\begin{tabular}{|l||c|c|}
\hline
        & Sans affichage de la triangulation & Zoom sur la triangulation \\
\hline
Image initiale & \includegraphics[width=5.5cm]{../images/image1.png} &  \includegraphics[width=5.5cm]{../images/image1_wire.png}\\
\hline
Simplification, 10 000 triangles & \includegraphics[width=5.5cm]{../images/image1_t10000.png} &  \includegraphics[width=5.5cm]{../images/image1_t10000_wire.png}\\
\hline
Simplification, 5 000 triangles & \includegraphics[width=5.5cm]{../images/image1_t5000.png} &  \includegraphics[width=5.5cm]{../images/image1_t5000_wire.png}\\
\hline
\end{tabular}
\end{center}

\begin{center}
\begin{tabular}{|c|c|}
\hline
\includegraphics[width=8cm]{../images/image1_t10000_wire_vue2.png} & \includegraphics[width=8cm]{../images/image1_t5000_wire_vue2.png}\\
Simplification, 10 000 triangles & Simplification, 5 000 triangles \\
\hline
\end{tabular}
\end{center}





\pagebreak
\subsection{Image2}
\begin{center}
\begin{tabular}{|l||c|c|}
\hline
        & Sans affichage de la triangulation & Zoom sur la triangulation \\
\hline
Image initiale & \includegraphics[width=5.5cm]{../images/image2.png} &  \includegraphics[width=5.5cm]{../images/image2_wire.png}\\
\hline
Simplification, 10 000 triangles & \includegraphics[width=5.5cm]{../images/image2_t10000.png} &  \includegraphics[width=5.5cm]{../images/image2_t10000_wire.png}\\
\hline
Simplification, 2 500 triangles & \includegraphics[width=5.5cm]{../images/image2_t2500.png} &  \includegraphics[width=5.5cm]{../images/image2_t2500_wire.png}\\
\hline
\end{tabular}
\end{center}

\begin{center}
\begin{tabular}{|c|c|}
\hline
\includegraphics[width=8cm]{../images/image2_t10000_wire_vue2.png} & \includegraphics[width=8cm]{../images/image2_t2500_wire_vue2.png}\\
Simplification, 10 000 triangles & Simplification, 2 500 triangles \\
\hline
\end{tabular}
\end{center}

\pagebreak
\subsection{Image3}
\begin{center}
\begin{tabular}{|l||c|c|}
\hline
        & Sans affichage de la triangulation & Zoom sur la triangulation \\
\hline
Image initiale & \includegraphics[width=5.5cm]{../images/tigre.png} &  \includegraphics[width=5.5cm]{../images/tigre_wire.png}\\
\hline
Simplification, 10 000 triangles & \includegraphics[width=5.5cm]{../images/tigre_t10000.png} &  \includegraphics[width=5.5cm]{../images/tigre_t10000_wire.png}\\
\hline
Simplification, 5 000 triangles & \includegraphics[width=5.5cm]{../images/tigre_t5000.png} &  \includegraphics[width=5.5cm]{../images/tigre_t5000_wire.png}\\
\hline
\end{tabular}
\end{center}

\begin{center}
\begin{tabular}{|c|c|}
\hline
\includegraphics[width=8cm]{../images/tigre_t10000_wire_vue2.png} & \includegraphics[width=8cm]{../images/tigre_t5000_wire_vue2.png}\\
Simplification, 10 000 triangles & Simplification, 5 000 triangles \\
\hline
\end{tabular}
\end{center}


\end{document}
